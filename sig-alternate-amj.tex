% This is "sig-alternate.tex" V2.1 April 2013
% This file should be compiled with V2.5 of "sig-alternate.cls" May 2012
%
% This example file demonstrates the use of the 'sig-alternate.cls'
% V2.5 LaTeX2e document class file. It is for those submitting
% articles to ACM Conference Proceedings WHO DO NOT WISH TO
% STRICTLY ADHERE TO THE SIGS (PUBS-BOARD-ENDORSED) STYLE.
% The 'sig-alternate.cls' file will produce a similar-looking,
% albeit, 'tighter' paper resulting in, invariably, fewer pages.
%
% ----------------------------------------------------------------------------------------------------------------
% This .tex file (and associated .cls V2.5) produces:
%       1) The Permission Statement
%       2) The Conference (location) Info information
%       3) The Copyright Line with ACM data
%       4) NO page numbers
%
% as against the acm_proc_article-sp.cls file which
% DOES NOT produce 1) thru' 3) above.
%
% Using 'sig-alternate.cls' you have control, however, from within
% the source .tex file, over both the CopyrightYear
% (defaulted to 200X) and the ACM Copyright Data
% (defaulted to X-XXXXX-XX-X/XX/XX).
% e.g.
% \CopyrightYear{2007} will cause 2007 to appear in the copyright line.
% \crdata{0-12345-67-8/90/12} will cause 0-12345-67-8/90/12 to appear in the copyright line.
%
% ---------------------------------------------------------------------------------------------------------------
% This .tex source is an example which *does* use
% the .bib file (from which the .bbl file % is produced).
% REMEMBER HOWEVER: After having produced the .bbl file,
% and prior to final submission, you *NEED* to 'insert'
% your .bbl file into your source .tex file so as to provide
% ONE 'self-contained' source file.
%
% ================= IF YOU HAVE QUESTIONS =======================
% Questions regarding the SIGS styles, SIGS policies and
% procedures, Conferences etc. should be sent to
% Adrienne Griscti (griscti@acm.org)
%
% Technical questions _only_ to
% Gerald Murray (murray@hq.acm.org)
% ===============================================================
%
% For tracking purposes - this is V2.0 - May 2012

\documentclass{sig-alternate-05-2015}

\def\sharedaffiliation{%
\end{tabular}
\begin{tabular}{c}}

\begin{document}

% Copyright
\setcopyright{acmcopyright}
%\setcopyright{acmlicensed}
%\setcopyright{rightsretained}
%\setcopyright{usgov}
%\setcopyright{usgovmixed}
%\setcopyright{cagov}
%\setcopyright{cagovmixed}


% DOI
\doi{10.475/123_4}

% ISBN
\isbn{123-4567-24-567/08/06}

%Conference
\conferenceinfo{PLDI '13}{June 16--19, 2013, Seattle, WA, USA}

\acmPrice{\$15.00}

%
% --- Author Metadata here ---
\conferenceinfo{WOODSTOCK}{'97 El Paso, Texas USA}
%\CopyrightYear{2007} % Allows default copyright year (20XX) to be over-ridden - IF NEED BE.
%\crdata{0-12345-67-8/90/01}  % Allows default copyright data (0-89791-88-6/97/05) to be over-ridden - IF NEED BE.
% --- End of Author Metadata ---

\title{superTLS: A Diverse and Redundant Secure Communication Channel for Privacy in Cloud}
%\subtitle{[Extended Abstract]
%\titlenote{A full version of this paper is available as
%\textit{Author's Guide to Preparing ACM SIG Proceedings Using
%\LaTeX$2_\epsilon$\ and BibTeX} at
%\texttt{www.acm.org/eaddress.htm}}}
%
% You need the command \numberofauthors to handle the 'placement
% and alignment' of the authors beneath the title.
%
% For aesthetic reasons, we recommend 'three authors at a time'
% i.e. three 'name/affiliation blocks' be placed beneath the title.
%
% NOTE: You are NOT restricted in how many 'rows' of
% "name/affiliations" may appear. We just ask that you restrict
% the number of 'columns' to three.
%
% Because of the available 'opening page real-estate'
% we ask you to refrain from putting more than six authors
% (two rows with three columns) beneath the article title.
% More than six makes the first-page appear very cluttered indeed.
%
% Use the \alignauthor commands to handle the names
% and affiliations for an 'aesthetic maximum' of six authors.
% Add names, affiliations, addresses for
% the seventh etc. author(s) as the argument for the
% \additionalauthors command.
% These 'additional authors' will be output/set for you
% without further effort on your part as the last section in
% the body of your article BEFORE References or any Appendices.

\numberofauthors{3} %  in this sample file, there are a *total*
% of EIGHT authors. SIX appear on the 'first-page' (for formatting
% reasons) and the remaining two appear in the \additionalauthors section.
%
\author{
% You can go ahead and credit any number of authors here,
% e.g. one 'row of three' or two rows (consisting of one row of three
% and a second row of one, two or three).
%
% The command \alignauthor (no curly braces needed) should
% precede each author name, affiliation/snail-mail address and
% e-mail address. Additionally, tag each line of
% affiliation/address with \affaddr, and tag the
% e-mail address with \email.
%
% 1st. author
\alignauthor
Andr\'e Joaquim\\
       \affaddr{INESC-ID, Instituto Superior T\'ecnico, Universidade de Lisboa}\\
       \affaddr{R. Alves Redol 9, 1000-029}\\
       \affaddr{Lisbon, Portugal}\\
% 2nd. author
\alignauthor
Miguel L. Pardal\\
       \affaddr{INESC-ID, Instituto Superior T\'ecnico, Universidade de Lisboa}\\
       \affaddr{R. Alves Redol 9, 1000-029}\\
       \affaddr{Lisbon, Portugal}\\
% 3rd. author
\alignauthor
Miguel Correia\\
       \affaddr{INESC-ID, Instituto Superior T\'ecnico, Universidade de Lisboa}\\
       \affaddr{R. Alves Redol 9, 1000-029}\\
       \affaddr{Lisbon, Portugal}\\
       %
       \end{tabular}
       \begin{tabular}{c}
       \email{\{andre.joaquim, miguel.pardal, miguel.p.correia\}@tecnico.ulisboa.pt}  
}
% use '\and' if you need 'another row' of author names
% There's nothing stopping you putting the seventh, eighth, etc.
% author on the opening page (as the 'third row') but we ask,
% for aesthetic reasons that you place these 'additional authors'
% in the \additional authors block, viz.
%\additionalauthors{Additional authors: John Smith (The Th{\o}rv{\"a}ld Group,
%email: {\texttt{jsmith@affiliation.org}}) and Julius P.~Kumquat
%(The Kumquat Consortium, email: {\texttt{jpkumquat@consortium.net}}).}
\date{03 May 2016}
% Just remember to make sure that the TOTAL number of authors
% is the number that will appear on the first page PLUS the
% number that will appear in the \additionalauthors section.

\maketitle
\begin{abstract}
We present superTLS, a diverse and redundant vulnerability-tolerant channel for privacy in cloud.
There have always been concerns about the strength of some of encryption mechanisms used in SSL/TLS channels and some of them were regarded as insecure at some point in time.
superTLS is our solution to mitigate the problem of secure communication channels being vulnerable to attacks due to unexpected vulnerabilities in its encryption mechanisms. superTLS' premise consists its mechanisms are vulnerable. superTLS relies on a combination of $k$ mechanisms/cipher suites, with $k$ being the diversity factor and $k > 1$.
Even when $k - 1$ mechanisms are regarded as insecure or considered vulnerable, superTLS relies on the remaining secure, diverse and redundant mechanism to maintain the channel secure.
We evaluated our channel by comparing it to a normal TLS channel and TLS-over-TLS.
\end{abstract}


%
% The code below should be generated by the tool at
% http://dl.acm.org/ccs.cfm
% Please copy and paste the code instead of the example below. 
%
\begin{CCSXML}
<ccs2012>
 <concept>
  <concept_id>10010520.10010553.10010562</concept_id>
  <concept_desc>Computer systems organization~Embedded systems</concept_desc>
  <concept_significance>500</concept_significance>
 </concept>
 <concept>
  <concept_id>10010520.10010575.10010755</concept_id>
  <concept_desc>Computer systems organization~Redundancy</concept_desc>
  <concept_significance>300</concept_significance>
 </concept>
 <concept>
  <concept_id>10010520.10010553.10010554</concept_id>
  <concept_desc>Computer systems organization~Robotics</concept_desc>
  <concept_significance>100</concept_significance>
 </concept>
 <concept>
  <concept_id>10003033.10003083.10003095</concept_id>
  <concept_desc>Networks~Network reliability</concept_desc>
  <concept_significance>100</concept_significance>
 </concept>
</ccs2012>  
\end{CCSXML}

\ccsdesc[500]{Computer systems organization~Embedded systems}
\ccsdesc[300]{Computer systems organization~Redundancy}
\ccsdesc{Computer systems organization~Robotics}
\ccsdesc[100]{Networks~Network reliability}


%
% End generated code
%

%
%  Use this command to print the description
%
\printccsdesc

% We no longer use \terms command
%\terms{Theory}

\keywords{Diversity; Redundancy; Security; Secure channels}

\section{Introduction}
%The \textit{proceedings} are the records of a conference.
%ACM seeks to give these conference by-products a uniform,
%high-quality appearance.  To do this, ACM has some rigid
%requirements for the format of the proceedings documents: there
%is a specified format (balanced  double columns), a specified
%set of fonts (Arial or Helvetica and Times Roman) in
%certain specified sizes (for instance, 9 point for body copy),
%a specified live area (18 $\times$ 23.5 cm [7" $\times$ 9.25"]) centered on
%the page, specified size of margins (1.9 cm [0.75"]) top, (2.54 cm [1"]) bottom
%and (1.9 cm [.75"]) left and right; specified column width
%(8.45 cm [3.33"]) and gutter size (.83 cm [.33"]).
%
%The good news is, with only a handful of manual
%settings\footnote{Two of these, the {\texttt{\char'134 numberofauthors}}
%and {\texttt{\char'134 alignauthor}} commands, you have
%already used; another, {\texttt{\char'134 balancecolumns}}, will
%be used in your very last run of \LaTeX\ to ensure
%balanced column heights on the last page.}, the \LaTeX\ document
%class file handles all of this for you.
%
%The remainder of this document is concerned with showing, in
%the context of an ``actual'' document, the \LaTeX\ commands
%specifically available for denoting the structure of a
%proceedings paper, rather than with giving rigorous descriptions
%or explanations of such commands.

\begin{itemize}
	\item {Description}
	\item The contributions of this paper are: x,y,z
	\item The rest of the paper is organized as followed: z,y,x
\end{itemize}

\section{Related Work}

What are the current issues/vulnerabilities in the existing systems/mechanisms? 

\begin{itemize}
	\item{Brief TLS}
	\item{Mechanisms vulnerabilities}
	\item{Brief combining mechanisms OR brief diversity}
\end{itemize}

\section{SuperTLS/Architecture}

\section{Experimental evaluation}

\section{Conclusions}
This paragraph will end the body of this sample document.
Remember that you might still have Acknowledgments or
Appendices; brief samples of these
follow.  There is still the Bibliography to deal with; and
we will make a disclaimer about that here: with the exception
of the reference to the \LaTeX\ book, the citations in
this paper are to articles which have nothing to
do with the present subject and are used as
examples only.
%\end{document}  % This is where a 'short' article might terminate

%ACKNOWLEDGMENTS are optional
%\section{Acknowledgments}
%This section is optional; it is a location for you
%to acknowledge grants, funding, editing assistance and
%what have you.  In the present case, for example, the
%authors would like to thank Gerald Murray of ACM for
%his help in codifying this \textit{Author's Guide}
%and the \textbf{.cls} and \textbf{.tex} files that it describes.

%
% The following two commands are all you need in the
% initial runs of your .tex file to
% produce the bibliography for the citations in your paper.
\bibliographystyle{abbrv}
\bibliography{sigproc}  % sigproc.bib is the name of the Bibliography in this case
% You must have a proper ".bib" file
%  and remember to run:
% latex bibtex latex latex
% to resolve all references
%
% ACM needs 'a single self-contained file'!
%
\section{References}
Generated by bibtex from your ~.bib file.  Run latex,
then bibtex, then latex twice (to resolve references)
to create the ~.bbl file.  Insert that ~.bbl file into
the .tex source file and comment out
the command \texttt{{\char'134}thebibliography}.
% This next section command marks the start of
% Appendix B, and does not continue the present hierarchy
%\section{More Help for the Hardy}
%The sig-alternate.cls file itself is chock-full of succinct
%and helpful comments.  If you consider yourself a moderately
%experienced to expert user of \LaTeX, you may find reading
%it useful but please remember not to change it.
%\balancecolumns % GM June 2007
% That's all folks!
\end{document}
