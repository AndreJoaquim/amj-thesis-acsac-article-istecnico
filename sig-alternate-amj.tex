% This is "sig-alternate.tex" V2.1 April 2013
% This file should be compiled with V2.5 of "sig-alternate.cls" May 2012
%
% This example file demonstrates the use of the 'sig-alternate.cls'
% V2.5 LaTeX2e document class file. It is for those submitting
% articles to ACM Conference Proceedings WHO DO NOT WISH TO
% STRICTLY ADHERE TO THE SIGS (PUBS-BOARD-ENDORSED) STYLE.
% The 'sig-alternate.cls' file will produce a similar-looking,
% albeit, 'tighter' paper resulting in, invariably, fewer pages.
%
% ----------------------------------------------------------------------------------------------------------------
% This .tex file (and associated .cls V2.5) produces:
%       1) The Permission Statement
%       2) The Conference (location) Info information
%       3) The Copyright Line with ACM data
%       4) NO page numbers
%
% as against the acm_proc_article-sp.cls file which
% DOES NOT produce 1) thru' 3) above.
%
% Using 'sig-alternate.cls' you have control, however, from within
% the source .tex file, over both the CopyrightYear
% (defaulted to 200X) and the ACM Copyright Data
% (defaulted to X-XXXXX-XX-X/XX/XX).
% e.g.
% \CopyrightYear{2007} will cause 2007 to appear in the copyright line.
% \crdata{0-12345-67-8/90/12} will cause 0-12345-67-8/90/12 to appear in the copyright line.
%
% ---------------------------------------------------------------------------------------------------------------
% This .tex source is an example which *does* use
% the .bib file (from which the .bbl file % is produced).
% REMEMBER HOWEVER: After having produced the .bbl file,
% and prior to final submission, you *NEED* to 'insert'
% your .bbl file into your source .tex file so as to provide
% ONE 'self-contained' source file.
%
% ================= IF YOU HAVE QUESTIONS =======================
% Questions regarding the SIGS styles, SIGS policies and
% procedures, Conferences etc. should be sent to
% Adrienne Griscti (griscti@acm.org)
%
% Technical questions _only_ to
% Gerald Murray (murray@hq.acm.org)
% ===============================================================
%
% For tracking purposes - this is V2.0 - May 2012

\documentclass{sig-alternate-05-2015}
% Maintain images and tables within their respective sections
\usepackage{caption} 

\graphicspath{ {images/} }

\def\sharedaffiliation{%
\end{tabular}
\begin{tabular}{c}}

\begin{document}

% Copyright
\setcopyright{acmcopyright}
%\setcopyright{acmlicensed}
%\setcopyright{rightsretained}
%\setcopyright{usgov}
%\setcopyright{usgovmixed}
%\setcopyright{cagov}
%\setcopyright{cagovmixed}

% DOI
\doi{10.475/123_4}

% ISBN
\isbn{123-4567-24-567/08/06}

%Conference
\conferenceinfo{PLDI '13}{June 16--19, 2013, Seattle, WA, USA}

\acmPrice{\$15.00}

%
% --- Author Metadata here ---
\conferenceinfo{WOODSTOCK}{'97 El Paso, Texas USA}
%\CopyrightYear{2007} % Allows default copyright year (20XX) to be over-ridden - IF NEED BE.
%\crdata{0-12345-67-8/90/01}  % Allows default copyright data (0-89791-88-6/97/05) to be over-ridden - IF NEED BE.
% --- End of Author Metadata ---

\title{SuperTLS: A Diverse and Redundant Secure Communication Channel for Privacy in Cloud}
%\subtitle{[Extended Abstract]
%\titlenote{A full version of this paper is available as
%\textit{Author's Guide to Preparing ACM SIG Proceedings Using
%\LaTeX$2_\epsilon$\ and BibTeX} at
%\texttt{www.acm.org/eaddress.htm}}}
%
% You need the command \numberofauthors to handle the 'placement
% and alignment' of the authors beneath the title.
%
% For aesthetic reasons, we recommend 'three authors at a time'
% i.e. three 'name/affiliation blocks' be placed beneath the title.
%
% NOTE: You are NOT restricted in how many 'rows' of
% "name/affiliations" may appear. We just ask that you restrict
% the number of 'columns' to three.
%
% Because of the available 'opening page real-estate'
% we ask you to refrain from putting more than six authors
% (two rows with three columns) beneath the article title.
% More than six makes the first-page appear very cluttered indeed.
%
% Use the \alignauthor commands to handle the names
% and affiliations for an 'aesthetic maximum' of six authors.
% Add names, affiliations, addresses for
% the seventh etc. author(s) as the argument for the
% \additionalauthors command.
% These 'additional authors' will be output/set for you
% without further effort on your part as the last section in
% the body of your article BEFORE References or any Appendices.

\numberofauthors{3} %  in this sample file, there are a *total*
% of EIGHT authors. SIX appear on the 'first-page' (for formatting
% reasons) and the remaining two appear in the \additionalauthors section.
%
\author{
% You can go ahead and credit any number of authors here,
% e.g. one 'row of three' or two rows (consisting of one row of three
% and a second row of one, two or three).
%
% The command \alignauthor (no curly braces needed) should
% precede each author name, affiliation/snail-mail address and
% e-mail address. Additionally, tag each line of
% affiliation/address with \affaddr, and tag the
% e-mail address with \email.
%
% 1st. author
\alignauthor
Andr\'e Joaquim\\
       \affaddr{INESC-ID, Instituto Superior T\'ecnico, Universidade de Lisboa}\\
       \affaddr{Lisbon, Portugal}\\
% 2nd. author
\alignauthor
Miguel L. Pardal\\
       \affaddr{INESC-ID, Instituto Superior T\'ecnico, Universidade de Lisboa}\\
       \affaddr{Lisbon, Portugal}\\
% 3rd. author
\alignauthor
Miguel Correia\\
       \affaddr{INESC-ID, Instituto Superior T\'ecnico, Universidade de Lisboa}\\
       \affaddr{Lisbon, Portugal}\\
       %
       \end{tabular}
       \begin{tabular}{c}
       \email{\{andre.joaquim, miguel.pardal, miguel.p.correia\}@tecnico.ulisboa.pt}  
}
% use '\and' if you need 'another row' of author names
% There's nothing stopping you putting the seventh, eighth, etc.
% author on the opening page (as the 'third row') but we ask,
% for aesthetic reasons that you place these 'additional authors'
% in the \additional authors block, viz.
%\additionalauthors{Additional authors: John Smith (The Th{\o}rv{\"a}ld Group,
%email: {\texttt{jsmith@affiliation.org}}) and Julius P.~Kumquat
%(The Kumquat Consortium, email: {\texttt{jpkumquat@consortium.net}}).}
\date{03 May 2016}
% Just remember to make sure that the TOTAL number of authors
% is the number that will appear on the first page PLUS the
% number that will appear in the \additionalauthors section.

\maketitle
\begin{abstract}
We present superTLS, a diverse and redundant vulnerability-tolerant secure communication channel for privacy in cloud.
There have always been concerns about the strength of some of encryption mechanisms used in SSL/TLS channels and some of them were regarded as insecure at some point in time.
SuperTLS is our solution to mitigate the problem of secure communication channels being vulnerable to attacks due to unexpected vulnerabilities in its encryption mechanisms. It is based on diversity and redundancy of cryptographic mechanisms and certificates to provide a secure communication channel even when one or more mechanisms are regarded vulnerable.
SuperTLS relies on a combination of $k$ mechanisms/cipher suites, with $k$ being the diversity factor and $k > 1$.
Even when $k - 1$ mechanisms are regarded as insecure or considered vulnerable, SuperTLS relies on the remaining secure, diverse and redundant mechanism to maintain the channel secure.
We evaluated the performance of our channel by comparing it to a normal TLS channel.
\end{abstract}

%
% The code below should be generated by the tool at
% http://dl.acm.org/ccs.cfm
% Please copy and paste the code instead of the example below. 
%
\begin{CCSXML}
<ccs2012>
<concept>
<concept_id>10003033.10003039.10003051</concept_id>
<concept_desc>Networks~Application layer protocols</concept_desc>
<concept_significance>500</concept_significance>
</concept>
<concept>
<concept_id>10003033.10003083.10003014</concept_id>
<concept_desc>Networks~Network security</concept_desc>
<concept_significance>300</concept_significance>
</concept>
<concept>
<concept_id>10010520.10010575.10010755</concept_id>
<concept_desc>Computer systems organization~Redundancy</concept_desc>
<concept_significance>500</concept_significance>
</concept>
<concept>
<concept_id>10002978.10003014.10003015</concept_id>
<concept_desc>Security and privacy~Security protocols</concept_desc>
<concept_significance>300</concept_significance>
</concept>
<concept>
<concept_id>10002978.10002979.10002982</concept_id>
<concept_desc>Security and privacy~Symmetric cryptography and hash functions</concept_desc>
<concept_significance>100</concept_significance>
</concept>
</ccs2012>
\end{CCSXML}

\ccsdesc[500]{Networks~Application layer protocols}
\ccsdesc[300]{Networks~Network security}
\ccsdesc[500]{Computer systems organization~Redundancy}
\ccsdesc[300]{Security and privacy~Security protocols}
\ccsdesc[100]{Security and privacy~Symmetric cryptography and hash functions}

%
% End generated code
%

%
%  Use this command to print the description
%
\printccsdesc

\keywords{Secure communication channels; Diversity; Redundancy; TLS; Vulnerability-Tolerance}

\section{Introduction}

\textit{Secure communication channels} are mechanisms that allow two entities to exchange messages or information securely in the Internet.
A secure communication channel has three properties: \textit{authenticity}, \textit{confidentiality}, and \textit{integrity}. Regarding authenticity, in an authentic channel, the messages can not be tampered. Regarding confidentiality, in a confidential channel, only the original receiver of a message is able to read that message. Regarding integrity, no one can impersonate another. The information regarding the original sender of a message can not be changed.
Several secure communication channels exist nowadays, such as TLS, IPsec or SSH. Each of these examples is used for a different purpose, but with the same finality of securing the communication.

\textit{Transport Layer Security (TLS)} is a secure communication channel widely used. Originally called Secure Sockets Layer (SSL), its first released version was SSL 2.0, released in 1995. SSL 3.0 was released in 1996, bringing improvements to its predecessor such as allowing forward secrecy and supporting SHA-1.
Defined in 1999, TLS did not introduce major changes. Although, the changes introduced were enough to make TLS 1.0 incompatible with SSL 3.0.
% In order to grant compatibility, a TLS 1.0 connection can be downgraded to SSL 3.0, which brought security issues.
TLS 1.1 and TLS 1.2 are upgrades to TLS 1.0 which brought some improvements such as mitigating CBC (cipher block chaining) attacks and supporting more block cipher modes of operation to use with AES.
TLS is divided in two sub-protocols, Handshake and Record, constituted by several mechanisms each.
The Handshake protocol is used to establish or re-establish a communication between a server and client. The Record protocol is used to process the sent and received messages.

\textit{Internet Protocol Security (IPsec)} is an Internet layer protocol that protects the communication at a lower level than SSL/TLS, which operates at the Application layer \cite{IPsec}.

\textit{Secure Shell (SSH)} is an Application layer protocol, such as SSL/TLS. SSH is a protocol used for secure remote login and other secure network services over an insecure network \cite{SSH}.

A secure communication channel becomes insecure when a vulnerability is discovered. Vulnerabilities may concern the protocol's specification, cryptographic mechanisms used by the protocol or specific implementations of the protocol. Many vulnerabilities have been discovered in SSL/TLS originating new versions of the protocol with renewed security aspects such as deprecating cryptographic mechanisms or enforcing security measures.
Concrete implementations of SSL/TLS have been also considered vulnerable by having implementation details causing a breach of security and affecting devices worldwide.

SuperTLS is a secure communication channel tolerant to vulnerabilities which does not rely on only one cryptographic mechanism. It is our belief that \textit{diversity} and \textit{redundancy} of cryptographic mechanisms and certificates can help mitigate existent vulnerabilities.
In our project's context, diversity and redundancy consist in using two or more different mechanisms/cipher suites with the same objective. For example, MD5 and SHA-3 are both hash functions used to generate digests. In a real case, where MD5 has become insecure, our diverse and redundant secure communication channel relies upon SHA-3 to keep the communication secure. Using diversity and redundancy of cryptographic mechanisms, when a one of those mechanisms is successfully attacked, another mechanism is able to maintain the security and availability of the communication.

\begin{itemize}
	\item Maybe talk a little about the project's context -- the cloud and communication between clouds
\end{itemize}

SuperTLS is part of an European project named SafeCloud...

Using diversity and redundancy, SuperTLS aims at increasing security over current secure communication channels by guaranteeing a diversity factor $k > 1$, which implies having $k$ different cipher suites with optimally $k$ different mechanisms for each of the following:
\begin{itemize}
\item Key exchange;
\item Authentication;
\item Encryption;
\item {MAC: HMAC/AEAD (used for data integrity). \textbf{NOTE: In TLS, HMAC is used for CBC and stream cipher. AEAD is used for GCM and CCM (MtE)}}
\end{itemize}

Although SSL/TLS supporting strong encryption mechanisms, such as AES and RSA, there are other factors than mathematical complexity that can contribute to vulnerabilities.
Diversifying encryption mechanisms includes diversifying certificates and consequently keys (public, private, shared).
Diversity of certificates is a direct consequence of diversifying encryption mechanisms due to the fact that each certificate is related to an authentication and key exchange mechanism.

The contributions of this paper are a new secure communication channel which uses diversity and redundancy to tolerate vulnerabilities in existent cryptographic mechanisms based on TLS 1.2 and specific evidence that diversity and redundancy can be employed without creating an excessive overhead in the communication. Proving that diversity and redundancy have a real impact on increasing security of a communication channel, while having reasonable performance and time-related costs, required a precise evaluation of our solution.

The rest of the document is organized in the following way: Section 2 presents the related work. Section 3 presents the architecture of the proposed solution. Section 4 presents the evaluation. Section 5 presents some conclusions.

\section{Related Work}
\label{sec-related-work}

What are the current issues/vulnerabilities in the existing systems/mechanisms? 

\begin{itemize}
	\item{Brief TLS}
\end{itemize}

\begin{itemize}
	\item{Mechanisms vulnerabilities}
	\item{Brief combining mechanisms OR brief diversity}
\end{itemize}

\section{Overview}

SuperTLS is a diverse and redundant vulnerability-tolerant secure communication channel. It aims at increasing security using diverse and redundant cryptographic mechanisms and certificates, and it is based on the TLS protocol. Although being an independent secure communication channel, SuperTLS is compatible with TLS 1.2.

This project aims to solve the main problem originated by having only one cipher suite negotiated between client and server: when one of the cipher suite's mechanisms becomes insecure, the communication channels using that cipher suite may become vulnerable.
Although most cipher suites' cryptographic mechanisms supported by TLS 1.2 are regarded as secure, as stated in Section \ref{sec-related-work}, there is not any assurance that a governmental agency or a company with high computational power and financial resources is not able to break one of those cryptographic mechanisms in the near future.

Unlike TLS, a SuperTLS communication channel does not rely in only one cipher suite. SuperTLS negotiates more than one cipher suite between client and server and, consequently, more than one cryptographic mechanism will be used for each \textbf{phase/purpose} -- key exchange, authentication, encryption and MAC.

Diversity and redundancy's first entry point in SuperTLS is in the SuperTLS' Handshake, where client and server negotiate $k$ cipher suites to be used in the communication, where $k$, $k\geq1$ is called the \textit{diversity factor}. In an abnormal case where the diversity factor $k = 1$, it is considered that the communication channel has no diversity nor redundancy. Nevertheless, in this case, SuperTLS works as a regular TLS 1.2 channel with one cipher suite.
The strength of SuperTLS resides in the fact that, even when $(k - 1)$ cipher suites become insecure, because one of its cryptographic mechanisms is insecure, our proposal remains invulnerable. The remaining secure, diverse and redundant cipher suite ensures that the communication channel is secured by remaining invulnerable.
The server chooses the best combination of $k$ cipher suites according to the cipher suites server and client have available. The choice of the cipher suites might be conditioned by the certificates of both server and client.
Diversity and redundancy will also naturally be introduced in the following communication between client and server. SuperTLS uses a subset of the $k$ cipher suites agreed-upon in the Handshake Protocol to encrypt the messages.

SuperTLS negotiates a default of $k = 2$ cipher suites, implying a diversity factor $k = 2$. While performance and management costs must be taken into account, this is the estimate of a reasonable $k$ cipher suites to use in the communication.

\textbf{NOTA1: Ver se a tese do ricardo diz algo acerca de k=2 ser o melhor}

\textbf{NOTA2: Ser\'a que posso afirmar que conforme o k cresce, a diversidade \'e cada vez menor e que 2 \'e a melhor relacao entre performance e diversidade?}

In terms of security, our solution must tolerate all the attacks given that at least one diverse redundant mechanism of the one attacked should not to be vulnerable that attack. On the contrary, our solution should never be vulnerable to attacks to which TLS 1.2 is not vulnerable.

\subsection{Protocol Specification}

SuperTLS's Handshake Protocol is similar to TLS Handshake Protocol. Messages' name are identical in order to provide easier migration and transition from TLS. Additionally, all the SuperTLS messages' names are analogous to TLS. Using this simplification, anyone who is familiarized with TLS can easily understand SuperTLS' Handshake messages and their purpose.

The messages which are diversity entry points are ClientHello, ServerHello, ServerKeyExchange, k-ServerKeyExchange, (Server and Client) Certificate, ClientKeyExchange and k-ClientKeyExchange.

The first message to be sent is called \textit{ClientHello}. This message's purpose is to inform the server that the client wants to establish a diverse secure channel for communication. The content of this first message consist in the client's protocol version, a Random structure (analogous to TLS 1.2) containing the current time and a 28-byte pseudo-randomly generated number, the session identifier, a list of the client's cipher suites and a list of the client's compression methods, if compression is to be used.

The server responds with a message named \textit{ServerHello}. ServerHello is a very important message as it is where the server sends to the client the $k$ cipher suites to be used in the communication. The server sends to the client its protocol version, a Random structure identical to the one sent by the client, the session identifier and the $k$ cipher suites chosen by the server from the list the client sent. It is also sent the compression method to use, if compression is to be used in the communication.

The server proceeds to send a \textit{(Server) Certificate} message containing its $k$ certificates to the client. The $k$ chosen cipher suites are dependent from the server's certificates. Each certificate is associated with one key exchange mechanism (KEM). Therefore, the $k$ cipher suites must use the key exchange mechanisms supported by the server's certificates.

\textbf{NOTA: Devo incluir o seguinte paragrafo?}

However, SuperTLS behaves normally if the server has $x, 0 < x < k$ certificates. The cipher suites to be used are chosen considering the available certificates. In this case, the diversity is not maximum due to the fact that a number of cipher suites will share the same certificate.

The \textit{ServerKeyExchange} message is the next message to be sent to the client by the server. This message is only sent if one of the $k$ cipher suites includes a key exchange mechanism which uses ephemeral keys, namely ECDHE or DHE. The contents of this message are the server's Diffie-Hellman ephemeral parameters. For every other $k - 1$ cipher suites using ECDHE or DHE, the server sends additional ServerKeyExchange messages with additional diverse Diffie-Hellman ephemeral parameters.

\textbf{NOTA: Devo tentar justificar porque \'e que mando mais mensagens em vez de tentar incluir tudo na mesma?}
Instead of computing all the ephemeral parameters and sending a larger message, the server computes one parameter at once and sends it immediately.

\begin{figure}[t]
\includegraphics[width=9cm]{superTLS-handshake}
\centering
\caption{SuperTLS' Handshake messages using a diversity factor $k$. The diversity and redundancy entry points are marked in blue and underlined.}
\label{fig:superTLS-example}
\end{figure}

%\begin{figure}[t]
%\includegraphics[width=9cm]{superTLS-example}
%\centering
%\caption{Example of SuperTLS' Handshake messages using $k = 2$ and choosing the combination TLS\_DHE\_DSS\_WITH\_AES\_128\_CBC\_SHA+\\TLS\_RSA\_WITH\_AES\_256\_CBC\_SHA256. The diversity and redundancy entry points are marked in blue and underlined.}
%\label{fig:superTLS-example}
%\end{figure}

The remaining messages sent by the server to the client at this point of negotiation, CertificateRequest and ServerHelloDone, are identical to TLS 1.2 \cite{TLS1.2-5246}.

\textbf{NOTA: O Client n\~ao precisa mesmo de enviar $k$ certificados... alterei no esquema. o cliente envia $i$ e o servidor envia $k$ ou $x$. Rever.}
The client proceeds to send a \textit{(Client) Certificate} message containing its $i$ certificates to the server, analogous to the (Server) Certificate message the client received previously from the server.

After sending its certificates, the client proceeds to send $k$ \textit{ClientKeyExchange} messages to the server. The content of these messages is based on the $k$ cipher suites chosen. If $m, 0 \leq m \leq k$ of the cipher suites use RSA as KEM, the client sends $m$ messages, each one with a RSA-encrypted pre-master secret to the server. If $j, 0 \leq j \leq k$ of the cipher suites use ECDHE or DHE, the client sends $j$ messages to the server containing its $j$ Diffie-Hellman public values. Even if a subset of the $k$ cipher suites share the same KEM, this methodology still applies as we introduce diversity by using different parameters for each cipher suite being used.

The server may need to verify the client's $i$ certificates. If they have signing capabilities, the client digitally signs all the previous handshake messages and sends them to the server for verification.

Client and server now exchange \textit{ChangeCipherSpec} messages, alike the Cipher Spec Protocol of TLS 1.2, in order to state that they are now using the previously negotiated cipher suites for exchanging messages in a secure fashion.

The client and server, in order to finish the Handshake, send to each other a \textit{Finished} message. This is the first message sent encrypted using the $k$ cipher suites negotiated earlier. Its purpose is to each party receive and validate the data received in this message. If the data is valid, client and server can now exchange messages over the communication channel.

\subsection{Combining diverse cipher suites}

Combining cryptographic mechanisms is not trivial as not all cryptographic mechanisms are compatible with one another. As referred in Section \ref{sec-related-work}, cryptographic mechanisms must be combined in a way that security is increased. For that to happen, we want to maximize diversity.
In order to fulfil these constraints, some research was made to determine and quantify diversity among some cryptographic mechanisms \cite{CarvalhoThesis14}.

Diversity is measured using different metrics for hash functions or public-key cryptographic functions. Hash functions' metrics include origin, year, digest size, structure, rounds and weaknesses (collisions, second preimage and preimage).
After comparing several hash functions using the metrics stated above, the authors concluded that the best three combinations are the following:
\begin{itemize}
\item {SHA-1 + SHA-3: This combination is not possible in SuperTLS. SHA-1 is regarded insecure and TLS 1.2 does not support SHA-3;}
\item {SHA-1 + Whirlpool: This combination is not possible in SuperTLS. SHA-1 is regarded insecure and TLS 1.2 does not support Whirlpool;} 
\item {SHA-2 + SHA-3: This combination is not possible in SuperTLS. TLS 1.2 does not support SHA-3.}
\end{itemize}

All the remaining suggested combinations cannot also be used because TLS 1.2 does not support SHA-3.
These results have a direct impact in SuperTLS due to the fact that, being SuperTLS based on OpenSSL, and compatible with TLS 1.2, it also does not support SHA-3 nor Whirlpool. All of SuperTLS' cipher suites use either AEAD (MAC-then-Encrypt mode using a SHA-2 variant) or SHA-2 (SHA-256 or SHA-384).

Having a small range of available hash functions limits the maximum diversity factor achievable concerning hash functions.
SHA-3 is relatively recent, having been selected the winner of the NIST hash function competition on 2012. In a near future, it is expected that a new TLS protocol version supports SHA-3 and makes possible the use of diverse hash functions.
Nevertheless, it still possible to achieve diversity by using different variants of SHA-2 -- SHA-256 and SHA-384.

Regarding public-key functions, the metrics used include origin, year, mathematical hard problems, perfect forward secrecy, semantic security and known attacks.
After comparing several public-key encryption mechanisms, using the metrics stated above, the authors concluded that the best four combinations are the following:
\begin{itemize}
\item {DSA + RSA: This combination is possible as TLS 1.2 suports both functions for \textit{authentication}. However, TLS 1.2 specific cipher suites only support DSA with elliptic curves (ECDSA);}
\item {DSA + Rabin-Williams: This combination is not possible. TLS 1.2 does not support Rabin-Williams;}
\item {RSA + ECDH: This combination is possible as TLS 1.2 supports both functions for \textit{key exchange};}
\item {RSA + ECDSA: This combination is possible as TLS 1.2 supports both functions for \textit{authentication}.}
\end{itemize}

Regarding authentication, although DSA + RSA is stated as the most diverse combination, TLS 1.2 preferred cipher suites use ECDSA instead of DSA. Using elliptic curves results in a faster computation and lower power consumption \cite{Gupta02}. With that being said, the preferred combination for authentication is RSA + ECDSA.

Regarding key exchange, the most diverse combination is RSA + ECDH. Although, in order to grant perfect forward secrecy, the ECDH must be employed using ephemeral keys (ECDHE). Concluding, the preferred combination for key exchange is RSA + ECDHE.

The study did not presented any conclusions regarding symmetric-key encryption, such as AES. Therefore, considering the metrics -- origin, year, and semantic security -- employed for public-key encryption functions, and considering an additional metric -- the mode of operation -- we obtained combinations of diverse symmetric-key encryption:
\begin{itemize}
\item {AES256-GCM + CAMELLIA128-CBC: This combination is possible as TLS 1.2 suports both functions;}
\item {AES256-CBC + CAMELLIA128-GCM: This combination is possible as TLS 1.2 suports both functions;}
\item {AES128-GCM + CAMELLIA256-CBC: This combination is possible as TLS 1.2 suports both functions;}
\item {AES128-CBC + CAMELLIA256-GCM: This combination is possible as TLS 1.2 suports both functions.}
\end{itemize}

Both AES and Camellia are supported by TLS 1.2 and are considered secure. The most diverse combination is AES256-GCM + CAMELLIA128-CBC. Their origin is different, they were first published in different years, they both have semantic security (as they both use initialization vectors) and the mode of operation is also different. One constraint of using this combination is that there is no cipher suite that uses RSA for key exchange, Camellia for encryption and a SHA-2 variant for MAC. Therefore, and in order to create maximum diversity, using Camellia implies using ECDHE for key exchange with Camellia.

As the cipher suites list is not very extensive, we are able to select a diverse group of $k$ cipher suites without generating a considerable amount of overhead. As the cipher suites are presented in order of preference, the first cipher suite chosen is, ideally, the first of the list.

The optimal combination of cipher suites is:\\
\textit{TLS\_RSA\_WITH\_AES\_256\_GCM\_SHA384} + \\\textit{TLS\_ECDHE\_ECDSA\_WITH\_CAMELLIA\_128\_CBC\_SHA256}.

For key exchange, as stated above, SuperTLS will use RSA and Ephemeral ECDH (ECDHE); for authentication, it will use RSA and Elliptic Curve DSA (ECDSA); for encryption, it will use AES-256 with Galois/Counter mode (GCM) and Camellia-128 with cipher block chaining (CBC) mode; finally, for MAC, it will use SHA-2 variants (SHA-384 and SHA-256).

Using this combination of cipher suites, maximum diversity is achieved using a diversity factor $k = 2$. The least diversified part of the communication is the MAC, due to the fact that TLS 1.2 does not support SHA-3 for now.

\subsection{Implementation}

\begin{itemize}
\item The server need to specify which certificates he wants to use by loading it to the SuperTLS
\item Compatible with TLS and OpenSSL but does not support all of its cipher suites.
\item I started by implementing diversity in the cipher.
\item The interface is the same as OpenSSL v1.0.2g, with the addition of SuperTLS specific functions
\item State my interface -- new functions (most important)
\end{itemize}

Functions:
\begin{itemize}
	\item{\textit{SSL\_CIPHER *ssl3\_choose\_sec\_cipher \\(SSL* s,\\ STACK\_OF(SSL\_CIPHER) *clnt,\\ STACK\_OF(SSL\_CIPHER) *srvr)}: Chooses a second cipher suite according to the second certificate and the first chosen cipher suite. This function tries to create maximum diversity.}
\end{itemize}

\section{Experimental evaluation}



\section{Conclusions}
This paragraph will end the body of this sample document.
Remember that you might still have Acknowledgments or
Appendices; brief samples of these
follow.  There is still the Bibliography to deal with; and
we will make a disclaimer about that here: with the exception
of the reference to the \LaTeX\ book, the citations in
this paper are to articles which have nothing to
do with the present subject and are used as
examples only.
%\end{document}  % This is where a 'short' article might terminate

%ACKNOWLEDGMENTS are optional
%\section{Acknowledgments}
%This section is optional; it is a location for you
%to acknowledge grants, funding, editing assistance and
%what have you.  In the present case, for example, the
%authors would like to thank Gerald Murray of ACM for
%his help in codifying this \textit{Author's Guide}
%and the \textbf{.cls} and \textbf{.tex} files that it describes.

\section{Future Work}

\begin{itemize}
\item Usar duas libraries compativeis e usar funcoes duma e doutra
\item Using SHA-3 for hash to increase diversity when it is supported by TLS
\end{itemize}

%
% The following two commands are all you need in the
% initial runs of your .tex file to
% produce the bibliography for the citations in your paper.
\bibliographystyle{abbrv}
\bibliography{sigproc}  % sigproc.bib is the name of the Bibliography in this case
% You must have a proper ".bib" file
%  and remember to run:
% latex bibtex latex latex
% to resolve all references
%
% ACM needs 'a single self-contained file'!
%
%\section{References}
%Generated by bibtex from your ~.bib file.  Run latex,
%then bibtex, then latex twice (to resolve references)
%to create the ~.bbl file.  Insert that ~.bbl file into
%the .tex source file and comment out
%the command \texttt{{\char'134}thebibliography}.
% This next section command marks the start of
% Appendix B, and does not continue the present hierarchy
%\section{More Help for the Hardy}
%The sig-alternate.cls file itself is chock-full of succinct
%and helpful comments.  If you consider yourself a moderately
%experienced to expert user of \LaTeX, you may find reading
%it useful but please remember not to change it.
%\balancecolumns % GM June 2007
% That's all folks!
\end{document}
